\chapter{Uvod}

Pojavom popularnih internetskih usluga za povezivanje korisnika stvorene su velike mreže društvenih zajednica. Generirane su velike količine podataka iz kojih je moguće izvući mnoštvo korisnih informacija. Takve zajednice sastoje se od puno manjih zajednica koje se po svojim karakteristikama razlikuju od ostalih. Zajednice je potrebno pronaći kako bi im se pristupilo na najbolji mogući način. Rješavanje ovog problema važno je i u drugim granama znanosti, kao na primjer u sociologiji, biologiji ili računarskoj znanosti, gdje su problemi predstavljeni na takav način, pomoću strukture grafa. Upravo su grafovi najpogodnija struktura podataka za pristup ovome problemu, gdje relevantne značajke, u primjeru društvenih mreža, ljude možemo prikazati pomoću čvorova, dok će bridovi predstavljati veze između tih značajki. Više bridova među određenim značajkama značit će da tu mogu postojati obilježja zajednice, npr. u biologiji bi to mogla biti tkiva koja u organima obavljaju sličnu ulogu. 

Rješavanje problema koji su predstavljeni grafovima je vrlo složeno, vremenski i prostorno. Ovakvi grafovi nisu jednostavnog oblika, ali u njima postoje određene pravilnosti, koje se mogu iskoristiti. U tu svrhu razvijeno je mnogo algoritama za otkrivanje zajednica koji različitim pristupima pokušavaju pronaći rješenje ovog problema. Pojedini algoritmi su bolji od drugih na jednom tipu društvenih mreža ili lošiji na drugom te se zato koriste evaluacijske mjere kojima se procjenjuje koliko je dobro rješenje koje je algoritam pronašao. Što više algoritama se testira s različitim društvenim mrežama i evaluacijskim mjerama dobit će se bolji uvid u to kada je koji bolje koristiti. Najpoznatiji algoritam među njima je Girvan-Newmanov algoritam koji će se nešto detaljnije opisati uz još nekoliko njih. Sličnim problemima bavili su se radovi Lancichinettija i Fortunata \cite{fortunato2016community} iz 2016. te \cite{lancichinetti2009community} iz 2009. godine.

Unatrag nekoliko posljednjih  godina uvedeni su zakoni o zaštiti osobnih podataka te je sada znatno teže dobiti pristup korisnim informacijama. Zato se koriste posebni algoritmi za generiranje umjetnih skupova podataka koji će za zadane parametre generirati grafovi pomoću kojih se mogu provoditi istraživanja. Kako bi generirani grafovi bili reprezentativni, moraju zadovoljavati svojstva karakteristična grafovima društvenih mreža koja se očituju u kratkoj udaljenosti između bilo koja dva čvora grafa te u visokom koeficijentu grupiranja. Mreže koje imaju navedena svojstva nazivaju se small-world mreže.


\begin{figure}
	\makebox[\textwidth][c]{\includegraphics[width=0.7\textwidth]{images/simple-community.png}}
	\caption{Primjer grafa nepreklapajućih društvenih zajednica. Izvor \cite{jayawickrama_2021}.}
	\label{fig:comm1}
\end{figure}


U sklopu diplomskog rada uspoređeno je pet algoritama za otkrivanje zajednica: Girvan-Newman, Louvain, Surprise, Leiden i Walktrap. Skupovi podataka korišteni pri usporedbi su umjetno generirane mreže prema Watts-Strogatz modelu i  primjeri stvarni društvenih mreža koji su dostupni kroz biblioteku SNAP. Pokazuje se da Girvan-Newmanov algoritam ima određene nedostatke u vidu algoritamske složenosti i značajno duljeg vremenskog izvršavanja što demonstriraju i rezultati evaluacijskih mjera. S druge strane, Louvain algoritam pokazuje vrlo dobra svojstva prema evaluacijskim mjerama te bi bio prikladan za upotrebu.

Ostatak rada organiziran je na sljedeći način. U poglavlju Društvene mreže i zajednice opisuju se struktura i karakteristična obilježja društvenih mreža i zajednica te način prikaza u računalu. Poglavlje Algoritmi otkrivanja društvenih zajednica opisuje teorijske detalje pojedinih algoritama i korake koje algoritmi obavljaju. U poglavlju Skupovi podataka opisani su korišteni primjeri društvenih mreža nad kojima su algoritmi pokretani. Poglavlje Programsko ostvarenje opisuje korištene biblioteke te implementaciju aplikacije kroz koju se provodila analiza. U poglavlju Vrednovanje i rezultati predstavljene su evaluacijske mjere i grafički su prikazani rezultati algoritama na testovima. Konačno, rad je zaključen u poglavlju Zaključak gdje se daje završni komentar o rezultatima usporedbe algoritama.
