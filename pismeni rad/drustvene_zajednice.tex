\chapter{Društvene mreže i zajednice}

Društvene mreže može se pronaći gdje god postoji sustav koji sadrži entitete koji su međusobno povezani. Primjera je mnogo, a neki od njih su: društvene web platforme, email mreža, web stranice koje sadrže poveznice prema dugima, uređaji koji su povezani preko internetske mreže i slično. 
Kako bi se skupina entiteta mogla nazvati društvenom zajednicom među njima mora postojati nekakav tip odnosa. Može biti jednosmjerni ili dvosmjerni te mogu postojati težine kojima se odnosu daje veća ili manja značajnost. Društvene mreže imaju složenu organizacijsku strukturu te se može pretpostaviti svojstvo lokalnosti koje kaže da ako jedan entitet ima veze prema neka druga dva entiteta onda je vjerojatnost da ta druga dva entiteta imaju vezu veća od prosječne. 

Društvene mreže imaju karakteristično svojstvo grupiranja u strukturu zajednice. Ako se čvorovi mreže mogu podijeliti u nepreklapajuće ili preklapajuće zajednice tako da broj veza između članova zajednice značajno premašuje broj veza između bilo koje dvije zajednice znači da mreža ima strukturu društvenih zajednica. Mreže koje imaju takvu strukturu često se mogu prikazati i kao hijerarijske strukture. U ovom radu obradit će se mreže koje sadrže nepreklapajuće strukture sa vezama koje nemaju određene težine.

Proces pronalaska društvenih zajednica jedan je od glavnih zadataka u analizama drušvenih mreža. Detekcija zajednica može biti vrlo korisna u raznim primjenama kao što je primjerice pronalaženje grupa kojima bi se mogle slati reklame za određene proizvode koji bi ih mogli zanimati umjesto da se svakom pojedincu šalju posebno. Još jedan primjer bio bi preporuka određenih sadržaja koji bi se mogli prikazivati grupama koje pokazuju zanimanja prema sličnim interesima. Primjera ima još mnogo, ali iz ova dva je već vidljivo da se korisne informacije mogu zaključivati iz društvenih mreža. Kako bi društvene mreže pohranili i analizirali u računalu potrebna je prikladna struktura podataka koja će u ovom slučaju biti graf.

Grafovi...