\chapter{Skupovi podataka}

U današnjem svijetu informacijski sustavi stvaraju goleme količine podataka. Podatke se može iskoristiti kako bi se poboljšali procesi koji se prate, pronašla područja u kojima postoji prostor za napredak, napraviti iskorak u poslovanju gdje se sustav primjenuje ili poboljšati korisničko iskustvo. Podaci su skupovi vrijednosti koji opisuju objekte u nekom procesu. Prate se kako bi se apstraktan proces iz ideje pretvorio u konkretne činjenice. Mogu se mjeriti, obrađivati i analizirati te vizualizirati kroz grafove, tablice i slike iz čega se dalje mogu izvoditi određeni zaključci. 


Algoritme koji se razvijaju da bi u podacima pronalazili korisne informacije potrebno je evaluirati te ocijeniti kako se ponašaju na kojem skupu podataka i pronaći situacije u kojima pokazuju najbolje performanse. Podaci prikupljeni iz stvarnih sustava najbolje opisuju praćene procese te je se najbolji zaključci mogu donijeti koristeći upravo njih.

Nakon problema i greški tijekom prikupljanja i praćenja osobnih podataka te loše sigurnosti i slučajeva krađe podataka, 2018. godine uvedena je uredba o općoj zaštiti podataka, poznata pod kraticom GDPR. Uredbom se kontrolira pohrana, prijenos i obrada osobnih podataka u Europskoj Uniji te su navedeni procesi znatno postroženi. Nakon Europske Unije slične odredbe primijenile su i neke američke savezne države te neke Azijske države čime odredba počinje vrijediti u gotovo svim razvijenijim dijelovima svijeta. Time je područje analiza društvenih mreža značajno pogođeno jer je postalo mnogo teže dobiti podatke o stvarnim korisnicima. Od tada značajnu ulogu počinju imati algoritmi za generiranje mreža koje imaju karakteristike društvenih zajednica. Algoritmi u početnom koraku dobivaju određene podatke o veličini i svojstvima željene mreže te generiraju takav primjer. U nastavku poglavlja opisat će se Watts-Strogatz model koji generira umjetne skupove podataka te stvarni skupovi podataka pomoću kojih će se algoritmi navedeni u radu testirati.

\section{Watts - Strogatz model}