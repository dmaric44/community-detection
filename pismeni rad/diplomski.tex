\documentclass[times, utf8, diplomski]{fer}
\usepackage{booktabs}
\usepackage{graphicx}
\usepackage{algorithm}
\usepackage{algorithmic}
\usepackage{url}
\def\UrlBreaks{\do\/\do-}

\begin{document}

\thesisnumber{2977}

\title{Usporedba algoritama otkrivanja zajednica u društvenim mrežama}

\author{Daniel Marić}

\maketitle

% Ispis stranice s napomenom o umetanju izvornika rada. Uklonite naredbu \izvornik ako želite izbaciti tu stranicu.
\izvornik

% Dodavanje zahvale ili prazne stranice. Ako ne želite dodati zahvalu, naredbu ostavite radi prazne stranice.
\zahvala{Zahvala mentoru prof.dr.sc Goranu Delaču na vodstvu i savjetima tijekom izrade diplomskog rada.}

\tableofcontents

\chapter{Uvod}

Pojavom popularnih internetskih usluga za povezivanje korisnika stvorene su velike mreže društvenih zajednica. Generirane su velike količine podataka iz kojih je moguće izvući mnoštvo korisnih informacija. Takve zajednice sastoje se od puno manjih zajednica koje se po svojim karakteristikama razlikuju od ostalih. Takve zajednice potrebno je pronaći kako bi im se pristupilo na najbolji mogući način. Rješavanje ovog problema važno je i u drugim granama znanosti kao na primjer u sociologiji, biologiji ili računarskoj znanosti gdje su problemi predstavljeni na takav način, pomoću strukture grafa. 

Upravo su grafovi najpogodnija struktura podataka za pristup ovome problemu gdje relevantne značajke, u primjeru društvenih mreža ljude, možemo prikazati pomoću čvorova dok će bridovi predstavljati veze između tih značajki. Više bridova među određenim značajkama značit će da tu mogu postojati obilježja zajednice, npr. u biologiji bi to mogla biti tkiva koja u organima obavljaju sličnu ulogu. 

Rješavanje problema koji su predstavljeni grafovima je vrlo složeno, vremenski i prostorno. Ovakvi grafovi nisu jednostavnog oblika, ali u njima postoje određene pravilnosti koje se mogu iskoristiti. U tu svrhu razvijeno je mnogo algoritama za otkrivanje zajednica koji rezličitim pristupima pokušavaju pronaći rješenje ovog problema. Pojedini algoritmi su bolji od drugih na jednom tipu društvenih mreža ili lošiji na drugom te se zato koriste evaluacijske mjere kojima se procjenjuje koliko je dobro rješenje koje je algoritam pronašao. Što više algoritama se testira s različitim društvenim mrežama i evaluacijskim mjerama dobit ćemo bolji uvid u to kada je koji bolje koristiti. Najpoznatiji algoritam među njima je Newman-Girvanov algoritam koji će se nešto detaljnije opisati uz još nekoliko njih. Sličnim problemima bavili su se radovi Lancichinettija i Fortunata \cite{fortunato2016community} iz 2016. te \cite{lancichinetti2009community} iz 2009. godine.

Unatrag posljednih nekoliko godina uvedeni su zakoni o zaštiti osobnih podataka te je sada znatno teže dobiti pristup korisnim informacijama. Zato se koriste posebni algoritmi za generiranje umjetnih skupova podataka koji će za zadane parametre generirati graf pomoću kojih se mogu provoditi istraživanja. 

U nastavku rada bit će opisana struktura i svojstva društvenih mreža i zajednica, algoritmi koji pronalaze društvene zajednice, skupovi podataka koji su korišteni u sklopu rada, programsko rješenje koje pokreće i evaluira rješenja algoritama te će se prikazati rezultati i do kojih se došlo. 


\begin{figure}
	\includegraphics[width=\linewidth]{images/simple-community.png}
	\caption{Primjer grafa nepreklapajućih društvenih zajednica.}
	\label{fig:comm1}
\end{figure}


\chapter{Društvene mreže i zajednice}

Društvene mreže može se pronaći gdje god postoji sustav koji sadrži entitete koji su međusobno povezani. Primjera je mnogo, a neki od njih su: društvene web platforme, email mreža, web stranice koje sadrže poveznice prema dugima, uređaji koji su povezani preko internetske mreže i slično. 
Kako bi se skupina entiteta mogla nazvati društvenom zajednicom među njima mora postojati nekakav tip odnosa. Može biti jednosmjerni ili dvosmjerni te mogu postojati težine kojima se odnosu daje veća ili manja značajnost. Društvene mreže imaju složenu organizacijsku strukturu te se može pretpostaviti svojstvo lokalnosti koje kaže da ako jedan entitet ima veze prema neka druga dva entiteta onda je vjerojatnost da ta druga dva entiteta imaju vezu veća od prosječne. 

Društvene mreže imaju karakteristično svojstvo grupiranja u strukturu zajednice. Ako se čvorovi mreže mogu podijeliti u nepreklapajuće ili preklapajuće zajednice tako da broj veza između članova zajednice značajno premašuje broj veza između bilo koje dvije zajednice znači da mreža ima strukturu društvenih zajednica. Mreže koje imaju takvu strukturu često se mogu prikazati i kao hijerarijske strukture. U ovom radu obradit će se mreže koje sadrže nepreklapajuće strukture sa vezama koje nemaju određene težine.

Proces pronalaska društvenih zajednica jedan je od glavnih zadataka u analizama društvenih mreža. Detekcija zajednica može biti vrlo korisna u raznim primjenama kao što je primjerice pronalaženje grupa kojima bi se mogle slati reklame za određene proizvode koji bi ih mogli zanimati umjesto da se svakom pojedincu šalju posebno. Još jedan primjer bio bi preporuka određenih sadržaja koji bi se mogli prikazivati grupama koje pokazuju zanimanja prema sličnim interesima. Primjera ima još mnogo, ali iz ova dva je već vidljivo da se korisne informacije mogu zaključivati iz društvenih mreža. Kako bi društvene mreže pohranili i analizirali u računalu potrebna je prikladna struktura podataka koja će u ovom slučaju biti graf.

\section{Reprezentacija društvenih mreža}
Prema definiciji jednostavan graf \textit{G} sastoji se od nepraznog konačnog skupa \textit{V(G)}, čije se elemente naziva vrhovi ili čvorovi grafa i konačnog skupa \textit{E(G)} različitih dvočlanih podskupova skupa \textit{V(G)} koji se naziva bridovima \cite{nakic_pavcevic_2019}. Graf može imati najviše $ {n(n-1) \over 2} $. U radu će se razmatrati jednostavni grafovi koji nemaju petlje i više bridova između istih čvorova. Bridovi će biti bestežinski i neusmjereni. 

Bitna definicija tiče se stupnja vrhova grafa. Stupanj vrha \textit{v} grafa \textit{G} je broj bridova koji su incidentni s \textit{v}. Stupanj vrha označava se sa \textit{deg(v)}. Vrh stupnja 0 zove se izolirani vrh, a vrh stupnja 1 krajnji vrh. \cite{nakic_pavcevic_2019}

Šetnja je graf sa skupom vrhova \textit{V(G) = $ \{x_{1},x_{2},...,x_{l}\} $ } i bridova \textit{E(G) = $ \{x_{0}x_{1},x_{1}x_{2},...,x_{l-1}x_{l}\} $ }. vrhovi $ x_{0} $ i $ x_{l} $ definiraju se kao krajevi dok je \textit{l} duljina šetnje. Ako su svi bridovi šetnje različiti tada se ona naziva staza. Ako su uz to i svi vrhovi različiti onda se takva šetnju naziva putem. Ako put počinje i završava u istom vrhu tada graf sadrži ciklus. Uz pretpostavljena ograničenja najmanji ciklus koji graf u ovom radu može imati je trokut što je često obilježje društvenih mreža.

Definicija puta omogućava definiranje važnog koncepta koji će se pojavljivati u radu pojedinih algoritama. Ako u grafu za svaki par vrhova postoji barem jedan put koji ide od jednog do drugog onda je graf povezan. Ako između vrhova postoji više putova onda je najkraći onaj koji ima najmanju duljinu. Promjer ili dijametar povezanog grafa je najveća udaljenost između bilo koja dva vrha u grafu. Ako ipak postoji barem jedan par vrhova između kojih ne postoji put onda je graf podijeljen u barem dva podgrafa. Svaki maksimalno povezani podgraf zove se komponenta povezanosti. Primjer se može vidjeti na slici \ref{fig:graph}.

\begin{figure}
	\makebox[\textwidth][c]{\includegraphics[width=0.7\textwidth]{images/nepovezani-graf.png}}
	%\includegraphics[width=0.7\textwidth]{images/nepovezani-graf.png}
	\caption{Primjer nepovezanog grafa}
	\label{fig:graph}
\end{figure}

Grafovi se mogu pohranjivati u obliku matrice susjedstva gdje su dva vrha, \textit{i} i \textit{j} susjedna ako im je element matrice \textit{A{ij}} jednak 1, a inače 0. Zbog pretpostavke da ne postoje petlje na dijagonali matrice susjedstva svi su elementi nule. Za reprezentaciju neusmjerenog grafa matrica susjedstva je simetrična što znači da je dovoljno pohraniti samo jedan trokut matrice, iznad ili ispod dijagonale. Suma elemenata \textit{i}-tog retka ili stupca jednaka je stupnju vrha \textit{i}

Jednostavniji oblik pohrane koji zauzima manje prostora u datoteci je takav da se pohranjuje popis bridova grafa te se i on može koristiti.


\section{Obilježja društvenih zajednica} 

Društvene zajednice moguće je definirati na nekoliko načina sa različitih stajališta, ali ne postoji niti jedna univerzalno prihvaćena definicija. Definiranje vrlo često ovisi o problemu koji se promatra zajedno sa specifičnim detaljima i primjenama gdje se pojam zajednice koristi. Prema radu \cite{fortunato2010community} zajednice je moguće promatrati iz lokalne i globalne perspektive.

Iz lokalne perspektive zajednica se može promatrati kao grupa entiteta koji su međusobno sličniji u odnosu na ostale entitete skupa podataka. Zajednica se formira tako što slični elementi imaju mnogo više interakcija sa članovima unutar zajednice u odnosu na one izvan. Zajednica se može smatrati kao autonomna skupina te ima smisla u određenim situacijama evaluirati svaku zasebno od ostatka društvene mreže. Stroga definicija društvene mreže kaže kako je društvena zajednica podgraf u kojem su svi članovi međusobno u interakciji \cite{luce1949method}. Takva definicija odgovara terminu klike u teoriji grafova koji označava skup vrhova koji su svi međusobno susjedni. Najjednostavniji primjer klike je trokut i oni se pojavljuju u svim društvenim mrežama. Veće klike od trokuta se pojavljuju rjeđe te ovakva definicija tako postaje manje praktična u stvarnim primjerima. Još jedan problem klike je to su tada svi vrhovi simetrični bez mogućnosti razlikovanja njihovih svojstava. U praktičnim primjerima očekuje se da među vrhovima postoji određena hijerarhijska struktura sa više i manje važnim čvorovima. Moguće je relaksirati pojam klike. Mogućnost je iskoristiti doseg i duljinu puta između čvorova. n-klika je takav podgraf da niti jedan par vrhova nije međusobno udaljen za više od \textit{n} koraka i skup je maksimalan u smislu da niti jedan drugi čvor nije udaljen za više od \textit{n} od svakog čvora iz podgrafa. Može se primijetiti da članovi podgrafa mogu biti povezani preko posrednika koji nije član grupe te onda n-klika ipak nije dovoljno dobra definicija. Definicija n-klana to popravlja. n-klan je n-klika u kojoj je dijametar podgrafa manji ili jednak \textit{n}. Takva definicija ima problem što u njoj i dalje postoji zahtjev n-klike te se tako dolazi do definicije n-kluba. n-klub je podgraf gdje je dijametar manji ili jednak \textit{n}. Tada je i svaki n-klan i n-klub i n-klika.

Iz globalne perspektive zajednica se može definirati promatrajući graf u cjelini. Takve definicije koriste se u slučajevima kada su zajednice dijelovi sustava bez kojih bi njegovo funkcioniranje bilo značajno. Definicije se najčešće izvode indirektno, iz algoritma prema kojem je neko svojstvo iskorišteno kako bi se zajednice otkrile. 

\section{Small-world mreže}

Small-world mreža je tip grafa u kojem većina čvorova nisu susjedi, ali susjedi nekog čvora imaju veliku vjerojatnost da su i oni susjedi te se do svakog čvora može doći kroz nekoliko koraka što znači da bilo koja dva čvora imaju kratku međusobnu udaljenost. Specifično je što se ona za dva slučajno izabrana čvora te za fiksiran prosječan stupanj vrha povećava proporcionalno logaritmu broj čvorova u grafu dok koeficijent grupiranja nije malen. Small-world mreže sadrže klike i grupe koje su gotovo klike što proizlazi iz visokog koeficijenta grupiranje. Društvene mreže posjeduju svojstva small-world mreže.

Koeficijent grupiranja je mjera stupnja u kojem čvorovi u grafu teže grupiranju. Postoje dvije verzije mjere, lokalna i globalna. U lokalnoj verziji mjera se računa za pojedini čvor te govori u kolikoj je on mjeri grupiran sa svojim susjedima. Mjera se za čvor \textit{i} računa kao suma broja veza koje postoje između susjeda promatranog čvora podijeljeno sa brojem svih mogućih veza, $ C_{i} = \dfrac{2 \mid e_{jk}:v_{j},v_{k} \in N_{i}, e_{jk} \in E \mid}{k_{i}(k_{i}-1)} $. Ako iz formule maknemo koeficijent 2 tada se ona može koristiti za usmjerene grafove.
Globalni koeficijent grupiranja daje informaciju o grupiranju u cijeloj društvenoj mreži. Temelji se na trojkama čvorova. Trojku čine promatrani čvor i druga dva čvora. Ako su povezani sa dva brida zovu se otvorena trojka, a ako su povezani sa tri zovu se zatvorena trojka što znači da jedan trokut čine tri trojke. Koeficijent se tada računa kao broj zatvorenih trojki podijeljen sa ukupnim brojem trojki, $ C = \dfrac{broj \; zatvorenih \; trojki}{ukupan \; broj \; trojki} $. Formula je primjenjiva i na usmjerene i neusmjerene grafove.

Kratka prosječna duljina puta između čvorova znači da postoje čvorovi sa velikim brojem veza odnosno visokim stupnjem. Takvi čvorovi nazivaju se sabirnice te služe kao posrednici u mnogim putevima između ostalih čvorova. Primjer iz stvarnog svijeta može se pronaći u zračnim letovima između gradova. Na putovanju između dva grada vrlo često nije potrebno više od tri leta jer mnogo letova ide preko jednog velikog grada sa puno letova prema drugima. 

Koliko mreža pripada small-world mreži može se izraziti pomoću small-koeficijenta, \textit{$\sigma$}, koji se računa tako da se uspoređuju koeficijent grupiranja i duljina puta u mreži sa slučajnim grafom koji ima jednak prosječan stupanj vrhova. Koeficijent se računa prema formuli:
\begin{equation}
 \sigma = \dfrac{\dfrac{C}{C_{r}}}{\dfrac{L}{L_{r}}}.
\end{equation}

$ C_{r} $ i $ L_{r} $ su mjera grupiranja i prosječna duljina puta u slučajnom grafu. Ako je $ \sigma $ > 1 tada se može smatrati da je mreža small-world. No mjera pokazuje lošu otpornost na rast broja čvorova u mreži.


\chapter{Algoritmi otkrivanja društvenih zajednica}
...555

\chapter{Skupovi podataka}

U današnjem svijetu informacijski sustavi stvaraju goleme količine podataka. Podatke se može iskoristiti kako bi se poboljšali procesi koji se prate, pronašla područja u kojima postoji prostor za napredak, napraviti iskorak u poslovanju gdje se sustav primjenuje ili poboljšati korisničko iskustvo. Podaci su skupovi vrijednosti koji opisuju objekte u nekom procesu. Prate se kako bi se apstraktan proces iz ideje pretvorio u konkretne činjenice. Mogu se mjeriti, obrađivati i analizirati te vizualizirati kroz grafove, tablice i slike iz čega se dalje mogu izvoditi određeni zaključci. 


Algoritme koji se razvijaju da bi u podacima pronalazili korisne informacije potrebno je evaluirati te ocijeniti kako se ponašaju na kojem skupu podataka i pronaći situacije u kojima pokazuju najbolje performanse. Podaci prikupljeni iz stvarnih sustava najbolje opisuju praćene procese te je se najbolji zaključci mogu donijeti koristeći upravo njih.

Nakon problema i greški tijekom prikupljanja i praćenja osobnih podataka te loše sigurnosti i slučajeva krađe podataka, 2018. godine uvedena je uredba o općoj zaštiti podataka, poznata pod kraticom GDPR. Uredbom se kontrolira pohrana, prijenos i obrada osobnih podataka u Europskoj Uniji te su navedeni procesi znatno postroženi. Nakon Europske Unije slične odredbe primijenile su i neke američke savezne države te neke Azijske države čime odredba počinje vrijediti u gotovo svim razvijenijim dijelovima svijeta. Time je područje analiza društvenih mreža značajno pogođeno jer je postalo mnogo teže dobiti podatke o stvarnim korisnicima. Od tada značajnu ulogu počinju imati algoritmi za generiranje mreža koje imaju karakteristike društvenih zajednica. Algoritmi u početnom koraku dobivaju određene podatke o veličini i svojstvima željene mreže te generiraju takav primjer. U nastavku poglavlja opisat će se Watts-Strogatz model koji generira umjetne skupove podataka te stvarni skupovi podataka pomoću kojih će se algoritmi navedeni u radu testirati.

\section{Watts - Strogatz model}

\chapter{Programsko ostvarenje}

Praktičan dio rada izveden je kroz desktop aplikaciju s grafičkim sučeljem. Aplikacija može pokrenuti ranije opisane algoritme i usporediti rezultate grupiranja prema evaluacijskim mjerama na odabranim skupovima podataka. Aplikacija je napisana u programskom jeziku Python \cite{van1995python}. Python je pogodan za rješavanje problema vezanih uz obradu i analizu podataka. Kroz njega je dostupno mnogo biblioteka koje su napisane za analizu specifičnih problema. Konkretne implementacije napravljene su u programskim jezicima kao što je C++ što značajno ubrzava izvođenje u odnosu kada bi implementacija bila napravljena u Pythonu. Nakon što se željeni algoritam izvrši, kroz Python se pružaju bogate mogućnosti u povezivanju s drugim bibliotekama koje se koriste i obradi dobivenih rezultata i vizualizaciji.

Za izradu grafičkog sučelja korišten je Pythonov paket tkinter. Za analizu rezultata i grafički prikaz korištena je biblioteka Matplotlib, dok su za rad s mrežama i njihovom analizom korištene tri biblioteke: SNAP, NetworkX i cdlib. Alati za analizu mreža moraju ispunjavati određene pretpostavke u radu. Moraju pružiti bogate funkcionalnosti za rad i obradu velikih mreža koje mogu imati milijune čvorova te implementirati algoritme koji će ih analizirati. Moraju biti u kompaktnom obliku kako bi memorijsko zauzeće bilo što manje	budući da su mnogi algoritmi ograničeni upravo memorijskim kapacitetima.

\section{Biblioteka SNAP}
Stanford Network Analysis Platform biblioteka (SNAP) \cite{leskovec2016snap} je sustav za analizu grafova i mrežnih sustava. Napisana je u programskom jeziku C++ te je optimizirana kako bi imala najbolje moguće performanse i na prikladan način predstavljala grafove. Biblioteka je osmišljena tako da su algoritmi koji se izvršavaju neovisni od tipa grafa ili mreže i njihove konkretne reprezentacije. Tako većina metoda radi za gotovo bilo koji tip grafa te je tim svojstvom dobivena mogućnost da se velike mreže, sa stotinama milijuna čvorova i milijardama bridova, dobro skaliraju. Kroz modul Snap.py većina SNAP funkcionalnosti dostupna je u programskom jeziku Python čime se olakšava njezino korištenje kroz napredne mogućnosti tog jezika. Za osnovne funkcionalnosti SNAP ne zahtjeva dodatne biblioteke. 

\begin{figure}
	\makebox[\textwidth][c]{\includegraphics[width=0.8\textwidth]{images/snap-slojevi.png}}
	\caption{Slojevi u dizajnu implementacije SNAP biblioteke \cite{leskovec2016snap}}
	\label{fig:SNAP_design}
\end{figure}

Implementacijski dizajn biblioteke podijeljen je u četiri sloja, što je prikazano na slici \ref{fig:SNAP_design}. U donjem sloju nalaze se klase skalara kao što su cijeli ili decimalni brojevi i stringovi. U njih se pohranjuju osnovni podaci o svakom vrhu. Iznad njega nalazi se sloj sa kompozitnim kolekcijama podataka kao što su vektori i hash tablice. One moraju efikasno pristupati pohranjenim elementima i iterirati kroz njih kako bi se obavljale operacije potrebne za rad algoritama. U sljedećem čvoru su klase koje su implementacije grafova te sadrže metode za održavanje strukture, odnosno dodavanje ili brisanje čvorova. Navedene metode moraju biti brze i učinkovite. Na vrhu se nalazi sloj sa metodama koji implementira algoritme i oslanja se na niže slojeve koji obavljaju operacije u pojedinim koracima.

Biblioteka se osim za izvor stvarnih primjera skupova podataka koristi i za pokretanje Girvan-Newman algoritma te kao rezultat vraća vrijednost modularnosti i pronađene zajednice. Poziva se sljedećom naredbom:
\begin{verbatim}
	modularity, communities = G.CommunityGirvanNewman().
\end{verbatim} 



\section{Biblioteka NetworkX}
NetworkX \cite{SciPyProceedings_11} je programska biblioteka jezika Python koja pruža alate za stvaranje, obradu i proučavanje strukture i ponašanja velikih mreža iz raznih područja primjene. Sadrži sučelje prema Pythonu i implementaciju brojnih tipova mreža i grafova kao što su jednostavni grafovi, usmjereni grafovi ili grafovi s paralelnim bridovima i petljama. Čvorovi mogu biti predstavljeni Python objektima koji implementiraju hash funkciju te mogu sadržavati proizvoljne podatke koji opisuju čvor.


Kompleksni algoritmi i numeričke operacije napisani su u jezicima C, C++ i FORTRAN. Biblioteka pruža mogućnosti rada sa raznim tipovima grafova, njihovom manipulacijom, konstrukcijom slučajnih modela te grafičkim prikazom grafova. Implementirani su algoritmi za računanje tipičnih svojstava grafa, npr. pronalaženje najkraćeg puta ili pronalaženja distribucije stupnjeva vrhova. Moguće je generirati mrežu sa small-world svojstvima prema Watts-Strogatz modelu na sljedećom naredbom, gdje su $N$ broj čvorova, $k$ broj susjeda i $p$ vjerojatnost prespajanja brida:
\begin{verbatim}
	G = nx.generators.watts_strogatz_graph(N, k, p).
\end{verbatim}
Biblioteka pruža potporu za rad sa raznim formatima ulaznih podataka te ako postoji ulazna datoteka koja sadrži graf zapisan pomoću liste susjedstva jednostavno se učitava na sljedeći način: 
\begin{verbatim}
	G = nx.read_adjlist(filename)
\end{verbatim}
Spremanje generiranog grafa u datoteku kao listu susjedstva izvršava se sljedećom naredbom:
\begin{verbatim}
	nx.write_adjlist(G, filename).
\end{verbatim}

Osim za pohranu grafova biblioteka NetworkX koristit će se za grafički prikaz manjih grafova pogodnih za vizualizaciju rješenja koje je algoritam pronašao. Graf sadrži redni broj čvora te su čvorovi različitih zajednica u različitim bojama. Primjer se može vidjeti na slici \ref{fig:drawing}. Iscrtavanje se poziva naredbom
\begin{verbatim}
	nx.draw(graph, pos = nx.spring_layout(graph), node_color=colorMap,
			with_labels=withLabels)
\end{verbatim}
kojoj se predaje graf, algoritam za razmještanje čvorova na zaslonu, mapa sa definiranim bojama za svaki čvor te logička varijabla kojom se uključuje ili isključuje oznake čvorova. 

\begin{figure}
	\makebox[\textwidth][c]{\includegraphics[width=0.8\textwidth]{images/draw-graph.png}}
	\caption{Grafički prikaz pronađenih zajednica u grafu Girvan-Newman algoritmom.}
	\label{fig:drawing}
\end{figure}

\pagebreak

\section{Biblioteka cdlib}
Community Discovery Library (cdlib) \cite{rossetti2019cdlib} je Python biblioteka za analizu i otkrivanje društvenih zajednica, stvorena na temelju grafovskih struktura podataka koje pružaju biblioteke NetworkX i Igraph. Biblioteka pruža implementacije raznih varijacija algoritama u području otkrivanje društvenih zajednica uključujući algoritme za pronalaženje nepreklapajućih zajednica, preklapajućih zajednica i neizrazitih zajednica gdje se za čvor računa razina kojom pripada zajednicama. Ukupno je implementirano 39 algoritama. Graf se definira preko strukture podataka koju nudi bilo koja od navedenih biblioteka te se nad njim pokreće algoritam iz cdlib biblioteke.

Biblioteka sadrži niz alata za usporedbu i evaluaciju kvaliteta pojednih grupa i čitavih rješenja koje algoritam pronalazi. Kada se izračunaju rješenja grupiranja za željenu mrežu tada cdlib omogućava evaluaciju koristeći mjere kvalitete, usporedbu sa alternativnim podjelama zajednica vizualizaciju rješenja za prikladne veličine grafova.

Iz cdlib biblioteke koristit će se implementacije za četiri algoritma detekcije zajednica: Louvain, Surprise, Leiden i Walktrap. Algoritmi se pokreću sljedećim naredbama:

\begin{verbatim}
	communities = algorithms.louvain(g_original)
\end{verbatim}


\begin{verbatim}
	communities = algorithms.surprise_communities(g_original)
\end{verbatim}

\begin{verbatim}
	communities = algorithms.leiden(g_original)
\end{verbatim}

\begin{verbatim}
	communities = algorithms.walktrap(g_original).
\end{verbatim}


Algoritmi kao rezultat vraćaju objekt razreda $NodeClustering$ koji sadržava informacije o pronađenim zajednicama, referencu na originalan graf, metapodatke o algoritmu koji se koristio, npr. ime algoritma i konfiguracijski parametri, zastavicu koja označava je li algoritam bio preklapajući ili nije te postotak čvorova koji su uključeni u grupiranje. Dobiveni objekt može se slati evaluacijskim funkcijama koje tada računaju rezultate mjera koje je algoritam pronađenom konfiguracijom zajednica ostvario.

\chapter{Vrednovanje i rezultati}
Rezultati.

\chapter{Zaključak}

Detekcija zajednica u društvenim mrežama važan je i kompleksan zadatak koji primjenu pronalazi u različitim društvenim i tehničkim znanostima. Društvenih mreža postoji mnogo, od bioloških i informatičkih sustava do socijalnih društvenih mreža, ali nije jednostavno doći do stvarnih primjera podataka koji su zaštićeni strogim zakonima o korisničkoj privatnosti. Zato su umjetno generirane mreže postale važan dio u istraživanju ovog područja analize podataka. U radu je opisan Watts-Strogatz model koji generira mreže sa small-world svojstvima što odgovara svojstvima društvenih mreža u stvarnom svijetu.

Girvan-Newmanov algoritam prvi je algoritam osmišljen za pronalazak društvenih zajednica u kompleksnim mrežama. Osim što je daleko najpoznatiji algoritam detekcije zajednica, nakon njegovog predstavljanja pokrenuo se značajan rast i napredak u analizi društvenih mreža. Na temelju njega nastavljen je rad u ovom području te su razvijeni mnogi drugi algoritmi kojima je cilj bio poboljšati rezultate i ispraviti nedostatke Girvan-Newmanovog algoritma. Kroz rad se pokazalo da algoritam ipak ima određene nedostatke u vidu složenosti na zahtjevnijim društvenim mrežama te kako je za veće sustave potrebno potražiti bolja rješenja.

Kako bi se ustanovilo koji algoritmi daju najbolja rješenja, osim primjera društvenih mreža potrebne su i kvalitetne evaluacijske mjere. Njima se mjeri kvaliteta rješenja koje je algoritam pronašao. Najpoznatija mjera je modularnost čiju vrijednost optimiziraju čak tri od pet opisanih algoritama. U radu je opisano i korišteno devet mjera kojima se s Girvan-Newmanovim algoritmom usporedilo Louvain, Surprise, Leiden i Walktrap algoritme. Svi algoritmi su značajno niže složenosti te se Louvain algoritam pokazao najuspješnijim na svim evaluacijskim mjerama na umjetno generiranim podacima i stvarnim primjerima društvenih mreža. Leiden i Surprise algoritmi nastali su kao poboljšane verzije Louvaina. Surprise algoritam pokušava umjesto modularnosti iskoristiti mjeru surprise kako bi mjerio kvalitetu zajednica dok zadržava ostale korake Louvain algoritma. Leiden algoritam uvodi dodatne korake kojima se pokušava popraviti kvaliteta pronađenih zajednica. Algoritmi ipak nisu pokazali bolja svojstva te su potrebna dodatna usavršavanja kako bi dostigli razinu Louvain algoritma. Walktrap algortam koristi slučajne šetnje u grafu kojima pokušava pronaći zajednice, ali je nešto veće složenosti od svih algoritama, osim Girvan-Newmana te nije pokazao značajno bolje rezultate od ranije navednih algoritama. U programskoj implementaciji korištene su biblioteke NetworkX i SNAP koje pružaju implementacije algoritama te imaju dostupne primjere stvarnih društvenih mreža.

Budući radovi mogu uključivati izvođenje algoritama na snažnijoj računalnoj opremi, proučavanje novih algoritama te analizu algoritama uključivanjem dodatnih evaluacijskih mjera. Vrlo važni postaju umjetni izvori podataka koji bi generiranjem mreža sa svojstvima što sličnijim mrežama iz stvarnog života pružili dodatne mogućnosti u analizi. Također, postoji prostor za razvoj samih algoritama gdje bi se primjenom heurističkih metoda ili drugih metoda optimizacije korištenjem lokalnih svojstava grafova skratilo vrijeme izvođenja algoritma te omogućilo lakšu i bržu analizu i detekciju zajednica u zahtjevnim mrežama s milijunima korisnika.


\bibliographystyle{fer}
\bibliography{literatura.bib}


\begin{sazetak}
U radu je opisano pet algoritama za otkrivanje zajednica u društvenim mrežama. Pored najpoznatijeg Girvan-Newmanovog algoritma opisani su Louvain, Surprise, Leiden i Walktrap algoritmi. Kao dio rada implementirana je programska podrška koja pruža mogućnosti usporedbe algoritama i generiranja umjetnih mreža. Rezultati algoritama vrednovani su pomoću devet evaluacijskih mjera na umjetno generiranim mrežama, dobivenim pomoću Watts-Strogatz modela te na primjerima mreža iz stvarnog svijeta iz biblioteke SNAP. 

\kljucnerijeci{društvene mreže, otkrivanje zajednica, Girvan-Newman, Louvain, Surprise, Walktrap, small-world svojstva, Watts-Strogatz model}
\end{sazetak}

\engtitle{Comparing community detection algorithms in social networks}
\begin{abstract}
Thesis describes five community detection algorithms for finding communities in social networks. In addition to the most famous Girvan-Newman algorithm, there are described Louvain, Walktrap, Leiden and Walktrap algorithms. As part of the thesis there is implemented software which provides ability to compare algorithms and generate artificial networks. The results of algorithms are evaluated with nine evaluation measures on artificial networks that are created using Watts-Strogatz model and on the real world network examples from the SNAP library.

\keywords{social networks, community detection, Girvan-Newman, Louvain, Surprise, Leiden, Walktrap, small-world properties, Watts-Strogatz model}
\end{abstract}

\end{document}
