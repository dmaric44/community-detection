\chapter{Vrednovanje i rezultati} \label{vrednovanja_i_rezultati}

Konačan cilj rada je vrednovati i usporediti algoritme za detekciju društvenih zajednica. U idealnom slučaju postojalo bi proizvoljno mnogo stvarnih primjera društvenih mreža nad kojima bi se algoritmi testirali, no zbog ranije opisanih problema teško je doći do novijih primjera. Time generirane društvene mreže postaju bitne te se koriste kako bi algoritmi bili testirani na što više primjera. Vrednovanje će se provesti nad nekoliko stvarnih primjera mreža i na umjetno stvorenim skupovima podataka. Kroz ovo poglavlje opisat će se korištene evaluacijske mjere, način provođenja testova nad različitim skupovima podataka te dobiveni rezultati kroz.

\section{Evaluacijske mjere}
Za reprezentativnu usporedbu algoritama potrebno je imati više kvalitetnih evaluacijskih mjera koje će pokazati prave odnose meuđu algoritmima nad različitim skupovima podataka. Mjere su osmišljene tako da koriste određena svojstva grafa u  pronađenim konfiguracijama zajednica kao što su stupnjevi vrhova u zajednicama ili bridovi koji se nalaze unutar i među zajednicama. Na temelju njih računaju mjeru koja se koristi za usporedbu. 


\subsection{Modularnost}
Modularnost je mjera kojom se procjenjuje odnos jakosti veza unutar zajednica i jakosti veza među zajednicama. Modularnost se računa prema sljedećoj formuli:
\begin{equation}
	Q = \frac{1}{2m} \sum_{u,v \in V} \bigg[ A_{u,v} - \frac{k_{u}k_{v}}{2m} \bigg] \delta(u,v).
\end{equation}

$m$ predstavlja ukupnu težinu bridova, odnosno broj bridova u bestežinskom grafu. Suma iterira kroz sve parove vrhova, $u$ i $v$. Vrijednost $k_{u}$ je suma svih težina bridova koji izlaze iz grafa, a u ovom slučaju bestežinskog neusmjerenog grafa će biti broj bridova koji su incidentni sa vrhom $u$. Isto vrijedi i za $k_{v}$. $A_{u,v}$ je težina brida između vrhova i iznosi 0 ako vrhovi nisu povezani ili je $u=v$. Funkcija $\delta(u,v)$ govori jesu li promatrani vrhovi u istoj zajednici ili nisu. Iznosi 1 ako jesu, a 0 ako nisu. Vrijednost modularnosti kreće se u intervalu $[ -\frac{1}{2}, 1 ]$. Ako je modularnost 0 ili manje znači da struktura podjele mreže nije jača od slučajne podjele vrhova u zajednice. U slučaju trivijalne podjele mreže u samo jednu zajednicu modularnost će iznositi 0. Što je iznos modularnosti veći to je podjela zajednica u mreži bolja \cite{brandes2007modularity}. Modularnost se koristi u tri od pet algoritama opisanih u radu. Girvan-Newman, Louvain i Leiden algoritam pokušavaju maksimizirati njezinu vrijednost. Vrijednost mjere izračunava se funkcijom definiranom u biblioteci NetworkX pozivom funkcije $nx.algorithms.community.modularity(graph, communities)$



\subsection{Tranzitivnost}
Tranzitivnost se definira kao prosječan koeficijent grupiranja čvorova u grafu. Izračunava se preko formule \ref{eq:triplets}, odnosno kao omjer broja zatvorenih trojki i ukupnog broja trojki u grafu. Tranzitivnost se može promatrati kao vjerojatnost pronalaženja izravne veze između dva vrha ako imaju zajedničkog susjeda. Vrijednost se kreću između 0 i 1. Trojke čvorova u grafovima društvenih mreža su vrlo česti te visoka vrijednost mjere ukazuje da zaista postoje društvene zajednice. No ako je vrijednost mjere niska ne mora nužno vrijediti da ne postoje strukture zajednica. Za izračunavanje mjere koristi se implementacija iz cdlib biblioteke te se prosječna vrijednost tranzitivnosti dobiva pozivom metode $evaluation.avg\_transitivity(graph, communities)$.


\subsection{Veličina zajednice}
Mjera veličina zajednica računa prosječnu veličinu zajednica u grafu. Mjera je vrlo jednostavna i ne otkriva mnogo o kvaliteti podjele zajednica, ali je koristan pokazatelj teži li algoritam pronalaženju većih ili manjih zajednica budući da se za umjetno generirane mreže zna kolike su zajednice veličine. Implementacija iz cdlib biblioteke poziva se naredbom $evaluation.size(graph,communities)$

\pagebreak
\subsection{Omjer vrhova koji sudjeluju u trokutu}
Mjera računa broj čvorova koji se pojavljuju kao dio trokuta u odnosu na ukupan broj čvorova u grafu. Vrijednosti mjere kreću se od 0 do 1 te bi u grafovima društvenih mreža trebale biti što više. Mjera se može iskazati sljedećom formulom:
\begin{equation}
	f(S) = \frac{| \{ u : u \in S, \{(u,v):v, w \in S (u,v) \in E, (u,w) \in E, (v,w) \in E \}  \notin \emptyset \} | }{n_{S}}
\end{equation}
Poziva se iz biblioteke cdlib naredbom \\ $evaluation.triangle\_participation\_ratio(graph, communities).$


\subsection{Prosječna ugrađenost vrhova}
Mjera ugrađenosti procjenjuje u kolikoj mjeri izravni susjedi promatranog vrha pripadaju istoj zajednici. Definirana je kao omjer unutarnjeg stupnja vrha i ukupnog stupnja vrha: 
\begin{equation}
	avg\_embd(c) = \frac{1}{|C|} \sum_{i \in C} \frac{k_{n}^{C}}{k_{n}}.
\end{equation}
Unutarnji stupanj vrha je broj bridova prema vrhovima koji su unutar iste zajednice kao i promatrani vrh. Maksimalna vrijednost ugrađenosti je 1 i postiže se kada su svi susjedi unutar iste zajednice, dok je minimalna vrijednost 0 i događa se u situaciji kada su svi susjedi vrha u nekoj drugoj zajednici. Izračun mjere poziva se naredbom iz biblioteke cdlib: $evaluation.avg\_embeddedness(graph,communities)$.


\subsection{Gustoća bridova unutar zajednica}
Mjerom se izražava koliki dio bridova u grafu povezuje vrhove unutar zajednica u odnosu na ukupan broj mogućih bridova grafa. Vrijednosti se kreću između 0 i 1 te što je mjera bliža 1 konfiguracija podjele zajednica je bolja. Mjera se može izraziti sljedećom formulom:
\begin{equation}
	f(S) = \frac{m_{S}}{\frac{n_{S}(n_{S} - 1)}{2}},
\end{equation}
gdje je $m_{S}$ broj bridova između vrhova unutar zajednica, a donji dio razlomka predstavlja izračun svih mogućih bridova grafa. Mjera se izračunava pozivom metode iz biblioteke cdlib, $evaluation.internal\_edge\_density(g,communities)$


\pagebreak
\subsection{Prosječan unutarnji stupanj}
Prosječan unutarnji stupanj društvene zajednice definira se kao omjer broja bridova unutar zajednice i ukupnog broja vrhova zajednice,
\begin{equation}
	f(S) =  \frac{2m_{s}}{n_{s}}.
\end{equation}
Maksimalna vrijednost mjere iznosi $n_{s} - 1$ u slučaju kada je zajednica potpuno poveza, a minimalna vrijednost će iznositi 0 u slučaju kada nema bridova između čvorova zajednice što upućuje na lošu strukturu zajednice. Mjera se izračunava pozivom metode $evaluation.average\_internal\_degree(g,communities)$ iz cdlib biblioteke.


\subsection{Surprise}
Mjera surprise detaljno je opisana u poglavlju \ref{surprise_alg} o Surprise algoritmu. Mjera je hipergeometrijska distribucija te pretpostavlja da se bridovi između vrhova pojavljuju prema određenoj vjerojatnosti. Osim kao dio algoritma, mjera se može iskoristiti i u evaluaciji gdje što je veći rezultat to je manja vjerojatnost slučajne konfiguracije promatrane zajednice, odnosno da je kvaliteta strukuture zajednice bolja. Izračun mjere obavlja se metodom $evaluation.surprise(graph, communities)$ iz biblioteke cdlib.


\subsection{Provodljivost}
Provodljivost je mjera kojom se iskazuje koliko jako je skup vrhova zajednice povezan s ostatkom mreže. Vrhovi izolirani od grafa imaju nisku vrijednost provodljivosti i čine kvalitetne zajednice. Mjera se može iskazati izrazom:
\begin{equation}
	f(S) = \frac{c_{S}}{2m_{S} + c_{S}},
\end{equation}
gdje je $c_{S}$ broj čvorova zajednice, a $m_{S}$ broj bridova unutar zajednice. Vrijednosti se kreću između 0 i 1 te je struktura zajednica bolja što je vrijednost provodljivost niža. Mjera se izračunava pozivom sljedeće metode iz biblioteke cdlib: $evaluation.conductance(graph, communities)$.


\pagebreak
\section{Postupak vrednovanje algoritama}
