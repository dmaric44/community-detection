\chapter{Zaključak}

Detekcija zajednica u društvenim mrežama važan je i kompleksan zadatak koji primjenu pronalazi u različitim društvenim i tehničkim znanostima. Društvenih mreža postoji mnogo, od bioloških i informatičkih sustava do socijalnih društvenih mreža, ali nije jednostavno doći do stvarnih primjera podataka koji su zaštićeni strogim zakonima o korisničkoj privatnosti. Zato su umjetno generirane mreže postale važan dio u istraživanju ovog područja analize podataka. U radu je opisan Watts-Strogatz model koji generira mreže sa small-world svojstvima što odgovara svojstvima društvenih mreža u stvarnom svijetu.

Girvan-Newmanov algoritam prvi je algoritam osmišljen za pronalazak društvenih zajednica u kompleksnim mrežama. Osim što je daleko najpoznatiji algoritam detekcije zajednica, nakon njegovog predstavljanja pokrenuo se značajan rast i napredak u analizi društvenih mreža. Na temelju njega nastavljen je rad u ovom području te su razvijeni mnogi drugi algoritmi kojima je cilj bio poboljšati rezultate i ispraviti nedostatke Girvan-Newmanovog algoritma. Kroz rad se pokazalo da algoritam ipak ima određene nedostatke u vidu složenosti na zahtjevnijim društvenim mrežama te kako je za veće sustave potrebno potražiti bolja rješenja.

Kako bi se ustanovilo koji algoritmi daju najbolja rješenja, osim primjera društvenih mreža potrebne su i kvalitetne evaluacijske mjere. Njima se mjeri kvaliteta rješenja koje je algoritam pronašao. Najpoznatija mjera je modularnost čiju vrijednost optimiziraju čak tri od pet opisanih algoritama. U radu je opisano i korišteno devet mjera kojima se s Girvan-Newmanovim algoritmom usporedilo Louvain, Surprise, Leiden i Walktrap algoritme. Svi algoritmi su značajno niže složenosti te se Louvain algoritam pokazao najuspješnijim na svim evaluacijskim mjerama na umjetno generiranim podacima i stvarnim primjerima društvenih mreža. Leiden i Surprise algoritmi nastali su kao poboljšane verzije Louvaina. Surprise algoritam pokušava umjesto modularnosti iskoristiti mjeru surprise kako bi mjerio kvalitetu zajednica dok zadržava ostale korake Louvain algoritma. Leiden algoritam uvodi dodatne korake kojima se pokušava popraviti kvaliteta pronađenih zajednica. Algoritmi ipak nisu pokazali bolja svojstva te su potrebna dodatna usavršavanja kako bi dostigli razinu Louvain algoritma. Walktrap algortam koristi slučajne šetnje u grafu kojima pokušava pronaći zajednice, ali je nešto veće složenosti od svih algoritama, osim Girvan-Newmana te nije pokazao značajno bolje rezultate od ranije navednih algoritama. U programskoj implementaciji korištene su biblioteke NetworkX i SNAP koje pružaju implementacije algoritama te imaju dostupne primjere stvarnih društvenih mreža.

Budući radovi mogu uključivati izvođenje algoritama na snažnijoj računalnoj opremi, proučavanje novih algoritama te analizu algoritama uključivanjem dodatnih evaluacijskih mjera. Vrlo važni postaju umjetni izvori podataka koji bi generiranjem mreža sa svojstvima što sličnijim mrežama iz stvarnog života pružili dodatne mogućnosti u analizi. Također, postoji prostor za razvoj samih algoritama gdje bi se primjenom heurističkih metoda ili drugih metoda optimizacije korištenjem lokalnih svojstava grafova skratilo vrijeme izvođenja algoritma te omogućilo lakšu i bržu analizu i detekciju zajednica u zahtjevnim mrežama s milijunima korisnika.