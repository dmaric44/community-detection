\chapter{Zaključak}

Detekcija zajednica u društvenim mrežama važan je i kompleksan zadatak koji primjenu pronalazi u različitim društvenim i tehničkim znanostima. Društvenih mreža postoji mnogo, od bioloških i informatičkih sustava do socijalnih društvenih mreža, ali nije jednostavno doći do stvarnih primjera koji su zaštićeni strogim zakonima o korisničkoj privatnosti. Zato su umjetno generirane mreže postale važan dio u istraživanju ovog područja analize podataka.

Girvan-Newman algoritamje prvi algoritam osmišljen za pronalazak društvenih zajednica u kompleksnim mrežama. Osim što je daleko najpoznatiji algoritam u ovom području, nakon njegovog predstavljanja pokrenuo se značajan rast i napredak u analizi društvenih mreža. Na temelju njega nastavljen je rad u ovom području te su razvijeni mnogi drugi algoritmi kojima je cilj bio poboljšati rezultate Girvan-Newmanovog algoritma. Kroz ovaj rad se pokazalo da algoritam ipak ima određene nedostatke u vidu složenosti na zahtjevnijim društvenim mrežama te kako je za veće sustave potrebno potražiti bolja rješenja.

Kako bi se ustanovilo koji algoritmi daju najbolja rješenja, osim primjera društvenih mreža potrebne su i kvalitetne evaluacijske mjere. Njima se mjeri kvaliteta rješenja koje je algoritam pronašao. U radu je opisano i korišteno devet mjera kojima se s Girvan-Newmanovim algoritmom usporedilo Louvain, Surprise, Leiden i Walktrap algoritme. Svi algoritmi su značajno niže složenosti te se Louvain algoritam pokazao  najuspješnijim na svim evaluacijskim mjerama. Leiden i Surprise algoritmi nastali su kao poboljšane verzije Louvaina. Surprise algoritam pokušava iskoristiti mjeru surprise kako bi mjerio kvalitete zajednica dok Leiden algoritam uvodi dodatne korake kojima se pokušava popraviti kvaliteta pronađenih zajednica. No algoritmi nisu pokazali bolja svojstva te su potrebna dodatna usavršavanja kako bi dostigli razinu Louvain algoritma. Walktrap algortam koristi slučajne šetnje u grafu kojima pokušava pronaći zajednice, ali je nešto veće složenosti od svih algoritama, osim Girvan-Newmana te nije pokazao značajno bolje rezultate od ranije navednih algoritama.

o pravim i umjetnim mrežama, i rad u budućnosti