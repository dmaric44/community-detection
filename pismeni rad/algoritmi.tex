\chapter{Algoritmi otkrivanja društvenih zajednica}

Ključan dio u pronalasku društvenih zajednica u društvenim mrežama su algoritmi koji ih otkrivaju. Oni moraju biti pouzdani i učinkoviti, ali se i izvršavati u prihvatljivom vremenskom okviru. Algoritmi se testiraju na brojnim skupovima podataka uz prikladne evaluacijske mjere kako bi se zaključilo u kojim uvjetima koji algoritam daje najbolje rješenje. 

Grafove koji predstavljaju društvene zajednice teško je prikazati u ravnini ako teže stvarnim veličinama koje se kreću u tisućama čvorova, a često i mnogo više, što znači da se ne može iz ljuske perspektive odrediti kako bi dobar raspored zajednica izgledao. To znači da su algoritmi koji pronalaze društvene zajednice nenadzirani algoritmi koji sami, bez primjera za učenje i unaprijednog znanja o njima pokušavaju pronaći rješenje. U društvenim mrežama algoritmi koriste topološke karakteristike i specifičnosti koje posjeduju ovakvi tipovi mreža. 

Dvije važne tehnike na kojima se temelji većina algoritama su particioniranje i grupiranje. Particioniranje grafova je proces u kojem se graf dijeli na unaprijed određeni broj manjih komponenti pomoću određenog svojstva. Svojstvo koje se može iskoristiti je minimalni rez. Ono se koristi tako da se graf podijeli na dva ili više razdvojenih podgrafova, a veličina reza koja se pokušava minimizirati je broj bridova koje je potrebno ukloniti da bi to ostvarili. Potrebno je odrediti i svojstvo koje bi odredilo veličinu komponenti kao primjerice minimalan ukupan stupanj vrhova kako bi se dobila rješenja koja imaju smisla. Zbog takvih zahtjeva ovakav pristup najčešće nije prihvatljiv jer broj zajednica nije moguće unaprijed odrediti.
Grupiranje je proces u kojem se entitete koji imaju zajedničke karakteristike svrstava u iste grupe. Pronalaženje grupa može dati informacije o skrivenim značajkama, vezama i svojstvima članova te koliko su međusobno čvrsto povezani. U hijerarhijskom grupiranju stvara se hijerarhija među zajednicama. Proces se može odvijati na dva načina, aglomerativni ili divizivni. U aglomerativnom načinu se koristi pristup koji ide od dna prema vrhu te se određeni čvor dodaje drugim sličnim čvorovima te se koristi određeni kriterij sličnosti. U divizivnom načinu veće grupe dijele se na manje uz korištenje određene mjere koja govori koliko je dobra trenutačna podjela prema kojoj će se odrediti konačan rezultat.



\section{Girvan-Newmanov algoritam}
Veliko zanimanje i rast aktivnosti znanstvene zajednice u području društvenih mreža potaknuo je rad Girvana i Newmana iz 2002. godine. 